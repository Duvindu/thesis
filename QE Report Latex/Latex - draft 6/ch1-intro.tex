%---------------------------------------------------------------------------------
\chapter{Introduction}
\label{chap:introduction}
%---------------------------------------------------------------------------------

\section{Background}
% background statistics on air travel growth
The explosive growth of traffic in the aviation industry has exposed existing systems and its related infrastructure to considerable strain. In Europe, an annual average of 9.7 million flights were recorded between 2006 to 2010 \cite{eurocontrol2011movements}. The forecasted number of flights in 2017 alone is expected to hit up to 12.5 million and by 2050, up to 27 million annually \cite{eurocontrol2013}. While the figures quoted are for the European continent, a similar trend can also be expected globally for annual air traffic volumes. Domestic air traffic in China alone is expected to grow 7.9\% annually between 2010 and 2030 \cite{secretariat2010icao}. In the same period between 2010 to 2030, air traffic growth is also forecasted elsewhere in Asia, with 6.7\% annual air traffic growth expected in South East Asia \cite{secretariat2010icao}.

With such an increase in air traffic, demand for data communications is also expected to swell. These demands stem not only from existing avionic systems but also from upcoming avionic systems and services which can be expected onboard aircrafts in near future. For instance, the newer generation of avionic systems can provide vital statistical information which can support real time health monitoring services to reduce aircraft maintenance time and to meet safety requirements. The newer systems can also include the provision of next generation in-flight entertainment services as well.

However, the deployment of new systems and services will increase demand for data communications. This is on top of providing adequate air traffic management (ATM) services to ensure that flight operations are not compromised. As a consequence, further strain is placed on existing Air-to-Ground (A/G) and Air-to-Air (A/A) aeronautical communication links, which are already operating under bandwidth constraints in a congested aeronautical spectrum. In addition, existing aeronautical communication links have also been noted to be inadequate in providing the needed capacity to handle the expected increases in data communications \cite{neji2013survey}. 

% increasing aeronautical spectral efficiency through channel splitting
Due to growing data communication demands, Very High Frequency (VHF) aeronautical communication systems (ACSs) have been relied upon to provide the necessary data communications capacity. In 2013, Aircraft Communications Addressing and Reporting System (ACARS) and VHF Data Link (VDL) were the most common ACSs in use worldwide \cite{neji2013survey}. Nonetheless, the incessant need to cope with current rising data capacity demands has resulted in various studies on improving current ACSs to increase spectral efficiency. Notably, the simplest of solutions to boost spectral efficiency is to increase the number of available channels by splitting existing channels into smaller portions \cite{jacobcognitive,stacey2008aeronautical}. However, in practical implementations, the actual number of VHF channels is lower due to the overlaying of 25KHz VDL channels as well as a lack of interest from other regional aeronautical hubs, e.g., in the United States. This is on top of introducing performance degradation to existing digital communication systems due to a constrained bandwidth \cite{neji2013survey}. As a result, the splitting of VHF channels into smaller 8.33KHz channels was implemented only in Europe. 

% increasing aeronautical spectral efficiency through improvements to VHF-based based ACSs
Apart from channel splitting, other works have also looked at VDL-based ACSs to increase data communication capacities. For instance, spectral resource management approaches for practical VDL-Mode 2 (VDL-2) ACSs were investigated by Ribeiro et al. \cite{ribeiro2014framework}. Through proper spectral resource management, it was found that the VDL-2 system can supprt a larger user capacity at the cost of higher overhead requirements, e.g., wider spectrum allocation and hand-off procedures. VDL-Mode 3 (VDL-3) and VDL-Mode 4 (VDL-4) have also been studied to potentially handle rising data capacity demands. For instance, the incurred overheads and latency due to delays and connection establishments in VDL-3 were studied for an ATN, i.e., A/G environment, in \cite{hung2000modeling}, \cite{hung2001enrouteModeling} while the capacity of VDL-4 systems was analysed in \cite{eurocontrol2010}. Specifically, it was noted that VDL-4 only performed similarly to VDL-2 under simulations. A comparison of the various VDL modes, e.g., VDL-2, VDL-3 and VDL-4, was studied by Bretmersky et al. \cite{bretmersky2002comparison} where it was found that despite being newer variants of VDL systems, the performance of both VDL-3 and VDL-4 does not justify the retirement of VDL-2 systems.

% increasing aeronautical spectral efficiency through improvements to B-MAC and ADS-B based ACSs
Aside from VDL-based ACSs, other approaches to meet rising data capacity demands have also been seen. A preliminary feasibility study to adopt Broadband-Aeronautical Mobile Communications (B-AMC), based on the OFDM concept, for use on the VHF spectrum was conducted by Lamiano et al. \cite{lamiano2009digital}. It was found that the VHF spectrum can be reassigned for en-route A/G data communications via B-AMC. However, this is also at a cost of wider VHF spectrum requirements, with 8 MHz required for B-AMC compared to the 2 MHz requirement for VDL-2. Tu et al. \cite{tu2008proposal} proposed the usage of the unallocated portion of the C band (5.09 GHz to 5.15 GHz) for aeronautical communications. Although the authors concluded that using the C band for aeronautical communications is feasible, communications in the C band is susceptible to the severe effects of Doppler shifts. An A/G ACS, based on the geolocation information from Automatic Dependent Surveillance - Broadcast (ADS-B), with adaptive modulation and beamforming capability was proposed by Nijsure et al. \cite{nijsure2016adaptive}. However, the proposed A/G communications system is susceptible to estimation errors and has a complex decision threshold computation process.

% short discussion on UAV systems and its impact on aeronautical spectral efficiency
The spectral efficiency of ACSs for unmanned aerial vehicles (UAVs) is also another area that must be considered. As noted in \cite{jacobcognitive}, \cite{kerczewski2013frequency} and \cite{mulkerin2007Lband}, UAV communication systems can communicate in the VHF/UHF, S, C or L band when the UAV is being piloted remotely. With the recent rise in popularity of UAVs for both military and commercial purposes, the UAV communication systems and the bands that these systems can access has only expedited the urgency of the spectral efficiency problem in aeronautical communications. This is because UAV communication systems must contend with ACSs (manned aerial vehicles) for the same communications spectrum. In addition, limitations to current aeronautical data communication systems have been noted \cite{neji2013survey}. For instance, VHF systems, such as ACARS and VDL, are susceptible to interference and are unable to cope with the increased data capacity demands. Furthermore, the improvements made to current ACSs have also been nominal \cite{neji2013survey}.

% motivation behind FCI program and the identification of future candidate technology. 
On this note, the Future Communications Infrastructure (FCI) program was launched by European Organization for the Safety of Air Navigation (EUROCONTROL) and the Federal Aviation Administration (FAA), USA. The FCI is a joint European-American program aimed at identifying possible candidate technologies and its respective infrastructures of global deployment for future aeronautical communications \cite{neji2013survey},\cite{eurocontrol2007communications}. 

In other words, the identified candidate technologies will enable the aviation industry to better cope with the scarcity of aeronautical spectral resources i.e. boosting spectral efficiency whilst providing adequate capacity for data communications on existing A/G and A/A links. Other similar research initiatives such as the Next Generation Air Transportation System (NextGen) and Single European Sky ATM Research (SESAR) supported by NASA \cite{wichgers2013study} and the European Commission \cite{BarbaSesar2011} respectively, are also being carried out in tandem and relative to the FCI program. Several candidate technologies have been identified by these research initiatives thus far. This includes Aeronautical Mobile Airport Communications System (AeroMACS) for airport tarmac communications and L-band Digital Aeronautical Communication System (LDACS) for continental A/G communications \cite{neji2013survey}.

\section{Motivations}
Although new candidate technologies have been singled out for possible use in future aeronautical communications, the issue of spectral efficiency still continues to plague the aviation industry. Problems such as the coexistence of, and interference caused, by future and legacy communication systems on a crowded aeronautical spectrum are just a subset of potential issues which must be dealt with. Therefore, tackling the spectral crunch facing the aviation industry is a crucial but necessary step that must be taken in due time. Tackling the spectral efficiency issue will enable aeronautical communication links to better meet future data capacity demands while managing competition for spectral resources between current and future aeronautical communication technologies more efficiently. The urgency of improving spectral efficiency is also not unique to the aviation industry alone and is a challenging research problem in various other industries related to wireless communication systems \cite{dai2015non}. 

The study of spectral efficiency has always been a traditional problem in the communications literature. Related discussions, especially in recent years, have generally tackled the spectral efficiency problem from either a spectral utilization perspective or through employing advanced signal processing algorithms to exploit diversity advantages. Studies related to efficient spectrum utilization approaches have looked at various associated technologies such as Cognitive Radio (CR), enhanced modulation techniques and In-Band Full Duplex (IBFD) radio systems. Many others have also applied spectral efficiency techniques that have been studied in literature to a wide range of fields such as smart grids \cite{kouhdaragh2013cognitive}, wireless sensor networks \cite{he2014cr} and Long Term Evolution (LTE) networks \cite{zulhasnine2010efficient}. 

In terms of spectral efficiency for aeronautical communications, the application of LTE for A/G communications has been proposed in \cite{AlcatelSrat}. Concepts from the communications literature, such as adaptive modulations, have also been adopted for aeronautical usage in the form of AeroMACS \cite{budinger2011aeronautical}. However, the lack of available spectral resources is an underlying problem hampering further development of aeronautical communications technology. Therefore, more can be done in this aspect to further boost spectral efficiency for aeronautical communications.

\section{Objective of Work}
It is clear from the earlier discussions that the communication systems of both manned and unmanned aerial vehicles must utilize the limited aeronautical spectrum efficiently. One way to tackle the aeronautical spectrum crunch is to transmit more data per channel usage. Although an adaptive modulation-based ACS has been proposed in the form of AeroMACS \cite{budinger2011aeronautical}, modulation schemes that are suitable for aeronautical communications can also be studied to improve the data rate per channel usage, which is explored in the current work. To this end, the major milestones pertaining to suitable modulation schemes for aeronautical communications are summarized below.

\subsection{Major Milestones: Modulation Schemes}
\begin{itemize}
	\item The Quad State-Paired QPSK (QS-PQPSK), proposed in \cite{tan2016quad} to improve the data rate of current A/G links, was simulated for various aeronautical communications scenarios. In particular, the bit error rate (BER) of the QS-QPSK was compared against differential 8 phase shift keying (D8PSK) under  combinations of Rician and Rayleigh fading where it was found that the former outperforms the latter.
	\item To further improve the BER performance of QS-PQPSK, a Space Time Block Coded QS-PQPSK (STBC QS-PQPSK) was proposed. The BER of the proposed STBC QS-PQPSK was compared against 8PSK, QS-PQPSK and Rotative QPSK (RQPSK), i.e., bench marked against modulation schemes carrying three bits per symbol. In a Rayleigh flat fading channel, the proposed STBC QS-PQPSK outperforms 8PSK, QS-PQPSK and RQPSK across the simulated SNR ranges. 
	\item When simulated in the various aeronautical communications scenarios, the proposed STBC QS-PQPSK also exhibited superior BER performance when compared against QS-PQPSK and D8PSK.
\end{itemize}

Another alternative to tackle the aeronautical spectrum crunch directly is to transit from half-duplex (HD) based ACSs to hybrid-duplex (HBD) or even full-duplex (FD), i.e., IBFD, based ACSs. HBD-ACSs consist of HD and FD nodes concurrently operating on the same spectrum while FD ACSs requires all nodes to operate in FD mode. Therefore, both HBD-based and FD-based ACSs can provide up to twice the spectral efficiency when compared to existing /legacy HD-based ACSs.

As a step towards transitioning from HD-based ACSs to FD-based ACSs, HBD-based ACSs can be considered to minimize potential disruption to the aviation community. In particular, the current work evaluates the performance of an HBD-ACS from the outage and finite signal-to-noise-ratio (SNR) diversity-multiplexing tradeoff (DMT) perspective. The outage and finite SNR DMT analysis is used to identify suitable operating scenarios for which the HBD-ACS outperforms existing /legacy HD-based ACSs. The major milestones pertaining to HBD-ACS for aeronautical communications are summarized below.

\subsection{Major Milestones: HBD-ACS}
\begin{itemize}
	\item The present work proposes an innovative approach for deriving closed-form expressions for outage probability for a II detector and a two-stage SIC detector in a Rician faded environment.
	\item It is shown that the proposed HBD-ACS attains superior outage performance over existing HD-ACS at low SNRs. At high SNRs, however, the outage performance of the proposed HBD-ACS is eclipsed by HD-ACS as the former becomes interference-limited at asymptotic SNRs. Nonetheless, we show through numerical simulations that the HBD-ACS can meet typical Quality-of-Service (QoS) requirements, e.g., frame error rate $\leq 10^{-3}$, at high SNRs for a range of interference levels through II and SIC detectors. 
	\item Closed-form finite SNR diversity gain expressions are derived for the II and SIC detectors under Rician fading. The asymptotic behavior of the derived finite SNR diversity gains for HBD-ACS and HD-ACS are proven and shown to be consistent with interference-limited outage behaviors at asymptotic SNRs.
	\item The proposed HBD-ACS is shown to achieve better diversity gains than HD-ACS at high multiplexing gains. Finite SNR DMT analysis reveals that operating at higher multiplexing gain causes the Rician $K$ factor, corresponding to the SOI, to have more impact on HBD-ACS outage performance. Additionally, reducing residual SI and interference from AS-1 leads to steeper decay of outage probability, improving the finite SNR DMT curve of the proposed HBD-ACS as a consequence.
\end{itemize}
 
\section{Report Outline}

The remaining part of this report is organized as follows.

The state-of-the-art and suitable spectral efficiency techniques for aeronautical communications are discussed in Chapter II. In particular, future candidate technologies for various flight domains in aeronautical communications are presented as part of the state-of-the-art. Studies on aeronautical channel modeling are then presented, before further discussions on suitable spectral efficiency techniques for aeronautical communications 

In Chapter III, improvements to aeronautical waveforms over aeronautical communication channels are discussed. In particular, a new modulation scheme is proposed, simulated and compared, over various fading environments, against existing modulation schemes used in aeronautical communications.

Discussions on a proposed HBD system for aeronautical communications are presented in Chapter IV. The performance of the proposed HBD system is evaluated from both the outage probability and finite SNR DMT perspective, and is compared against existing HD systems.

Finally, future extensions of the works in Chapter III and Chapter IV are presented before the conclusion of this report in Chapter V.

%
%an overview of aeronautical communication links and systems in Chapter II. Discussions in Chapter II will span VHF-band (Very High Frequency-band), L-band and C-band systems for both A/G and A/A communication links. Channel access methods will also be briefly mentioned. We will then present suitable spectral efficiency techniques and its respective state-of-the-art that can be adopted for aeronautical communications in Section III. Finally, potential research topics will be deliberated in Section IV before the conclusion of this paper in Section V.
%
%This report is organized as follows:
%
%Chapter 2 provides ....................................
%
%Chapter 3 reviews ....................................
%
%Chapter 4 discusses ....................................
%
%In Chapter 5, we propose ....................................
%
%Finally, we conclude in Chapter 6, where we also discuss about the
%directions and schedule of our future research.
