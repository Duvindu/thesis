\chapter* {Abstract}
\addcontentsline{toc}{section}{\numberline{}\hspace{-0.35in}{\bf
Abstract}}  % Add the Abstract to the table of contents using the specified format

As a result of booming air traffic growth in the near future, spectral efficiency in aeronautical communications, for both manned and unmanned aerial vehicles, is a pressing issue which the aviation community must address in due time. To begin the first step in addressing spectral efficiency in aeronautical communications, discussions on the state-of-the-art in aeronautical communications literature is presented, with candidate communications technologies noted for various flight domains. However, the identified candidate technologies do not directly address the lack of available aeronautical spectrum which prohibits spectral efficiency improvements. To this end, a literature survey of spectral efficiency techniques which are suitable for aeronautical communications is conducted, with discussions on potential adaptation of the discussed techniques for aviation as  research opportunities in aeronautical communications. 

Following the literature survey, Space Time Block Coded Quad State-Paired QPSK (STBC QS-PQPSK), based on the Quad State-Paired QPSK (QS-PQPSK) modulation, was proposed for Air-to-Ground (A/G) communications to improve spectral efficiency. Simulations comparing the proposed STBC QS-PQPSK against QSPQPSK and differential 9 phase shift keying (D8PSK) revealed that STBC QS-PQPSK has better BER performance against the other techniques. Thus, STBC QS-PQPSK is a suitable alternative for an efficient and reliable aeronautical waveform. 

To directly address the spectrum crunch faced by the aviation industry, a hybrid-duplex aeronautical communication system (HBD-ACS) consisting of a full-duplex (FD) enabled ground station (GS), and two half-duplex (HD) air-stations (ASs) is proposed. Closed-form outage probability and finite signal-to-noise ratio (SNR) diversity gain expressions over Rician fading channels are derived for a successive interference cancellation (SIC) detector. Similar expressions are also presented for an interference ignorant (II) detector and HD-equivalent modes at GS and ASs. Through outage and finite SNR diversity gain analysis conducted at the nodes, and system level, residual SI and inter-AS interference are found to be the primary limiting factors in the proposed HBD-ACS. Additional analysis also revealed that the II and SIC detectors in the proposed HBD-ACS are suitable for weak and strong interference scenarios, respectively. When compared to HD-ACS, the proposed HBD-ACS achieves lower outage probability and higher diversity gains at higher multiplexing gains when operating at low SNRs. Finite SNR analysis also showed the possibility of the proposed HBD-ACS being able to attain interference-free diversity gains through proper management of residual SI. Hence, the proposed HBD-ACS is more reliable and can provide better throughput compared to existing HD-ACS at low-to-moderate SNRs.


