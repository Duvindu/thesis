\chapter* {Abstract}
\addcontentsline{toc}{section}{\numberline{}\hspace{-0.35in}{\bf
Abstract}}  % Add the Abstract to the table of contents using the specified format

Spectrum scarcity is a significant challenge in aeronautical communications due to increasing demand for future wireless services in manned and unmanned aerial vehicle (UAV) networks. To this end, hybrid-duplex (HBD) systems are proposed as a pragmatic solution as it improves spectrum efficiency in aeronautical communications over various realistic environments.

Towards highlighting the feasibility of HBD aeronautical networks, this thesis first studies the outage probability and diversity gain of aeronautical networks with single uplink and downlink nodes for various interference management approaches. Then, the performance of HBD UAV networks is characterized in environments with fading and shadowing. Next, performance analysis is conducted for multi-UAV networks with an arbitrary number of uplink and downlink UAVs using stochastic geometry tools. Finally, we examine correlated non-orthogonal multiple access (NOMA) transmissions for multi-antenna receivers in UAV communication systems (UCSs). Through the various performance analysis, the benefits of HBD systems over half-duplex (HD) systems at low signal-to-noise ratio (SNR) regimes are highlighted. The major contributions of this thesis are further elaborated below. 

Firstly, for aeronautical networks with single uplink and downlink nodes, we present new closed-form expressions for outage probability and finite SNR analysis. Through extensive analysis, we show that HBD systems attain lower outage probability and higher diversity gain over conventional HD systems at low SNR regimes. Furthermore, we identify ideal scenarios that enable HBD systems to outperform HD systems for various interference management strategies.

Secondly, for HBD UAV networks experiencing fading and shadowing, we derive new closed-form expressions for the probability density function (PDF), cumulative distribution function (CDF), and fractional moments under Rician shadowed fading. We demonstrate that shadowing on the self-interference (SI) link has a negligible impact on the full-duplex (FD) ground station (GS), i.e., FD-GS. On the contrary, severe shadowing on the desired link leads to higher outage probability at the FD-GS, despite SI mitigation measures. At the downlink UAV, interference management through the joint detection approach is shown to be more robust to shadowing than the interference ignorant method.

Thirdly, for multi-UAV networks with an arbitrary number of uplink and downlink UAVs, we propose exact analytical frameworks that incorporate stochastic geometry tools, i.e., binomial point processes (BPPs), that accurately models the random location of UAVs. For FD heterogeneous networks (FD-HetNets) and HBD multi-UAV networks, we observe a lower outage probability, higher ergodic capacity, and greater number of deployed UAVs on the same spectrum, as compared to HD-based multi-UAV networks. Even with weaker SI suppression and strong oscillator phase noise at the FD-GS, FD-HetNets still attain higher ergodic sum capacity over the HD-HetNets. 

Finally, for multi-antenna receivers in NOMA-aided UCSs, we present new closed-form expressions for the joint PDF and joint CDF under the bivariate Rician shadowed fading model. An analysis of the NOMA-aided UCS demonstrates that more UAVs can be deployed on the same spectrum, at similar outage probability, than orthogonal multiple access based (OMA-based) UCSs. Also, cross correlation is shown to affect the diversity gain of both NOMA and OMA transmissions only at low SNR regimes.

Therefore, the analysis conducted in this thesis demonstrates the feasibility and benefits of addressing spectrum scarcity in aeronautical communications through HBD systems.


