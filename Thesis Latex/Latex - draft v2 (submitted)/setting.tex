%----------------------------------------------------------------------------
% File          PpHd.tex
% Author        Ong Kok Leong
% Description   preamble: the global settings
%-----------------------------------------------------------------------------

% 1. Define your report format here
\documentclass[12pt,a4paper,twoside]{report}

% 2. Specify the packages to be included
\usepackage{amssymb,amsmath,epsfig,fancyheadings,cite,float,dblfloatfix,appendix}
%\usepackage[dvips]{graphicx}
\usepackage[caption=false,font=footnotesize]{subfig}
\usepackage{graphicx}
\usepackage{xcolor}
\usepackage{rotating}
\usepackage{chngcntr}
\usepackage[multiple]{footmisc}
\usepackage[hyphens]{url}

\oddsidemargin  +0.10in %
\evensidemargin +0.00in %
\topmargin      +0.15in %
\textheight     +8.50in %
\textwidth      +6.25in %

\pagestyle{plain}

\font\Bold=cmbx10 scaled \magstep3
\def\EndOfProof{\nolinebreak\ \hfill\rule{1.5mm}{2.7mm}}
\def\endOfProof{\ \hfill\rule{1.5mm}{2.7mm}}

\def\proof{\noindent \textbf{Proof:}\quad}

\newcounter{theorem}
\newtheorem{theorem}{Theorem}
\renewcommand{\thetheorem}{\thechapter.\arabic{theorem}}

\newcounter{lemma}
\newtheorem{lemma}{Lemma}
\renewcommand{\thelemma}{\thechapter.\arabic{lemma}}

\newcounter{corollary}
\newtheorem{corollary}{Corollary}
\renewcommand{\thecorollary}{\thechapter.\arabic{corollary}}

\newcounter{proposition}
\newtheorem{proposition}{Proposition}
\renewcommand{\theproposition}{\thechapter.\arabic{proposition}}

\newcounter{observation}
\newtheorem{observation}{Observation}
\renewcommand{\theobservation}{\thesubsection.\arabic{observation}}

\newcounter{remark}
\newtheorem{remark}{Remark}
\renewcommand{\theremark}{\thechapter.\arabic{remark}}

\newcounter{result}
\newtheorem{result}{Result}
\renewcommand{\theresult}{\thesubsection.\arabic{result}}

\newcounter{summary}
\newtheorem{summary}{Summary}
\renewcommand{\thesummary}{\thesection.\arabic{summary}}

\renewcommand{\figurename}{Fig.}

\newcounter{mytempeqncnt}
\newcommand\eqa{\stackrel{(a)}{=}}
\newcommand\eqb{\stackrel{(b)}{=}}
\newcommand\eqaequal{\stackrel{(a)}{=}}
\newcommand\eqbequal{\stackrel{(b)}{=}}
\newcommand\eqcequal{\stackrel{(c)}{=}}


\counterwithin*{theorem}{chapter}
\counterwithin*{corollary}{chapter}
\counterwithin*{lemma}{chapter} 
\counterwithin*{proposition}{chapter}
\counterwithin*{observation}{subsection}
\counterwithin*{remark}{chapter}
\counterwithin*{result}{subsection}
\counterwithin*{summary}{section}

%\renewcommand{\thefigure}{\thechapter.\arabic{figure}}
%\renewcommand{\thesubfigure}{\thefigure.\alph{subfigure}}
%\makeatletter
%\renewcommand{\@thesubfigure}{\thesubfigure:\space}
%\renewcommand{\p@subfigure}{}

%\def\LEQ.#1.#2.#3{#1\!\leqslant\!#2\!\leqslant\!#3}
%\def\GEQ.#1.#2.#3{#1\!\geqslant\!#2\!\geqslant\!#3}
%\def\emptyset{\varnothing}
%\renewcommand{\theenumi}{\rm \roman{enumi}}
%\renewcommand{\labelenumi}{\rm (\theenumi)}
%\renewcommand{\theenumii}{\rm \alph{enumii}}
%\renewcommand{\labelenumii}{\rm (\theenumii)}
%
%\renewcommand{\topfraction}{0.75}
%\renewcommand{\bottomfraction}{0.75}
%\renewcommand{\textfraction}{0.10}

%
% Report specific definitions
%

%
% Definition in the context of this report.
%
%\def\reporttitle    {An Algorithmic Framework for Web Data Mining}
%\def\supp           {\varphi}
%\def\psupp          {\varphi^{\mathcal{P}}}
%\def\conf           {\delta}
%\def\pconf          {\delta^{\mathcal{P}}}
%\def\tab            {\hspace{0.8cm}}

%
% Define the set of symbols for the constraint set.
%
%\def\map            {\Pi}
%\def\gen            {\Upsilon}
%\def\prune          {\Phi}
%\def\selcand        {\Psi}
%\def\compute        {\Theta}
%\def\check          {\Omega}

%
% The set of operators
%
%\def\select         {\sigma}
%\def\proj           {\pi}
%\def\join           {\bowtie}
%\def\apply          {\mu}
%\def\undo           {\eta}
%\def\rename         {\leftarrow}
%
%\def\linespaceNormal        {\baselineskip=24pt}
%\def\linespaceTight         {\baselineskip=21pt}
%\def\linespaceVeryTight     {\baselineskip=19pt}

\renewcommand{\bibname}{References}

% Define some new macros
%\def\addnotation #1: #2#3{$#1$\indent \parbox{5in}{#2 \dotfill  \pageref{#3}}}
\def\addnotation #1: #2#3{$#1$ & #2 & \pageref{#3}}
\def\newnot#1{\label{#1}}
